%%%%%%%%%%%%%%%%%%%% author.tex %%%%%%%%%%%%%%%%%%%%%%%%%%%%%%%%%%%
%
% sample root file for your "contribution" to a contributed volume
%
% Use this file as a template for your own input.
%
%%%%%%%%%%%%%%%% Springer %%%%%%%%%%%%%%%%%%%%%%%%%%%%%%%%%%


% RECOMMENDED %%%%%%%%%%%%%%%%%%%%%%%%%%%%%%%%%%%%%%%%%%%%%%%%%%%
\documentclass[graybox]{svmult}

\usepackage[utf8]{inputenc}

% choose options for [] as required from the list
% in the Reference Guide

\usepackage{type1cm}        % activate if the above 3 fonts are
                            % not available on your system
%
\usepackage{makeidx}         % allows index generation
\usepackage{graphicx}        % standard LaTeX graphics tool
                             % when including figure files
\usepackage{multicol}        % used for the two-column index
\usepackage[bottom]{footmisc}% places footnotes at page bottom


\usepackage{newtxtext}       % 
\usepackage{newtxmath}       % selects Times Roman as basic font

% see the list of further useful packages
% in the Reference Guide

\makeindex             % used for the subject index
                       % please use the style svind.ist with
                       % your makeindex program

%%%%%%%%%%%%%%%%%%%%%%%%%%%%%%%%%%%%%%%%%%%%%%%%%%%%%%%%%%%%%%%%%%%%%%%%%%%%%%%%%%%%%%%%%

\begin{document}

\title*{Behavioral Analysis of Fuzzy Cognitive Maps}
% Use \titlerunning{Short Title} for an abbreviated version of
% your contribution title if the original one is too long
\author{Miklós F. Hatwagner}
% Use \authorrunning{Short Title} for an abbreviated version of
% your contribution title if the original one is too long
\institute{Miklós F. Hatwagner \at Széchenyi István University, Győr, Hungary \email{miklos.hatwagner@sze.hu}}
%
% Use the package "url.sty" to avoid
% problems with special characters
% used in your e-mail or web address
%
\maketitle

\abstract*{<To be prepared>}

\abstract{<To be prepared>}

\section{Introduction}
\label{sec:intro}

Decision support systems and methods are at the center of 
researcher's attention for a long time \cite{busemeyer1999dynamic}, 
because prudent decision making can be really hard in practice, 
especially if the effect of many related factors with continuously 
changing states have to be taken into account, and a wrong 
intervention may cause serious personal injury or damage to property.

Fuzzy Cognitive Maps (FCMs) \cite{b.kosko1986} can be one of the 
possible tools of a decision maker \cite{papageorgiou2013fuzzy}. 
They are bipolar fuzzy graphs that are made of nodes interconnected 
by directed, weighted edges. The various factors, variables of a 
complex system can be described by the nodes (also called \emph
{concepts}) and the strength of causal relations among them can be 
expressed by edges. Even if an FCM model can characterize the 
studied system only qualitatively \cite{j.l.salmeron2009}, this 
method is easy to use and provides a transparent, clear description 
of it. Modeling capabilities and simulations that reveal the dynamic 
behavior of the system made FCM an inevitable opportunity in 
decision support.

There are two main ways of model creation \cite{papageorgiou2012learning}:
\begin{enumerate}
  \item an expert, or a group of experts can give the description of 
  the system, or
  \item based on a long time-series database a suitable machine 
  learning technique may generate the model with more or less support 
  of experts.
\end{enumerate}

In the first case the expert who makes the model unavoidably 
includes his/her beliefs and subjectivity in the map. This can be 
decreased if a group of experts works out the model instead of a 
sole expert \cite{kosko1988hidden,groumpos2010fuzzy}, but the result 
will contain some uncertainty for sure. They can agree relatively 
quickly and easily on the number of concepts, the existence and sign 
of relationships, but the magnitude of those relationships are much 
harder to define, especially if an order of importance or strength 
exists among them. The number of relationships is a quadratic 
function of the number of concepts, therefore even if the modelers 
follow Kosko's advice and they do not include self-loops in the FCM, 
in case of a small model containing only 10 concepts the number of 
relationships can be up to 90. It will be really hard to define so 
many weights properly and worse is that the steepness parameter 
$\lambda$ of the threshold function does not have any connection 
with real objects, but its value may strongly influence the results 
of simulations (number and values of fixed points, limit cycles, 
chaotic behavior).

That is why the second method, based on machine learning is preferred 
in most cases. In those cases it can be applied the concepts 
are still defined by experts, however, and/or the nature of time series data. 
Sometimes such data is not available and only the first method 
remains.

No matter how the model is made, it worth investigating the effect 
of small modifications of connection weights or steepness parameter 
on model behavior. The sensitive points of the model can be revealed 
and experts can discuss about the possibilities on making the model 
more stable and reliable. Unfortunately it cannot be made by hand, 
following a trial and error approach. In an FCM the connection 
weights are specified by real numbers, thus the number of possible 
modified weights are practically infinite. With a reasonable 
compromise we can say that the weights to try can be limited to the 
number of used linguistic variables, eg. 5: -1, -0.5, 0, 0.5 and 1. 
In the already mentioned model containing only 10 concepts, where 
even 90 relationships may exist, the number of model versions may up 
to $5^{90} = 8,078\times10^{62}$. Obviously the experts are primarily 
interested only in the effect of slight modifications, thus they 
would be satisfied with the check of one less and one greater values 
(or keeping the current one), but it could still mean $3^{90} = 
8,728\times10^{42}$ model versions. That can be even worse if they 
want to take the effect of different $\lambda$ values into account and 
of course, one simulation is not enough to characterize the behavioral 
effect of a modification, many simulations (depending on the model 
size to cover all interesting parts of the search space) are needed 
and these simulations are rather time consuming on their own. There is 
a clear need for an automated, consistent method for such 
investigations, and the work on that started in 
\cite{hatwagner2016uncertainty,hatwagner2017behavioral} and further 
developed in \cite{hatwagner2019banking,hatwagner2018improved}. This 
task is too big even for a computer: an exhaustive search cannot be 
performed, but an evolutionary search technique, eg. the Bacterial 
Evolutionary Algorithm is able to guide the directions of search for a 
slightly modified model that behaves significantly different.

\input{references}
\end{document}
